\documentclass[12pt,t]{beamer}
\usepackage{graphicx}
\usepackage{tikz}
\setbeameroption{hide notes}
\setbeamertemplate{note page}[plain]
\usepackage{listings}

\input{header.tex}

%%%%%%%%%%%%%%%%%%%%%%%%%%%%%%%%%%%%%%%%%%%%%%%%%%%%%%%%%%%%%%%%%%%%%%
% end of header
%%%%%%%%%%%%%%%%%%%%%%%%%%%%%%%%%%%%%%%%%%%%%%%%%%%%%%%%%%%%%%%%%%%%%%

% title info

\title{Automating Reproducibility}
\subtitle{A Reproducible Data Analysis Workflow with R\nobreakspace{}Markdown, Git, Make, and Docker}
\author{\href{https://github.com/aaronpeikert/}{Aaron Peikert\textsuperscript{1,2}\\ \&\\ Andreas M. Brandmaier\textsuperscript{1,3}}}
\institute{
\textsuperscript{1}Max Planck Institute of Human Development\\
\textsuperscript{2}Humboldt{\textendash}Universität zu Berlin\\
\textsuperscript{3}Max Planck UCL Centre for Computational Psychiatry and Ageing Research
}
\date{
\scriptsize {\lolit Slides:} \href{https://github.com/aaronpeikert/repro-talk}{\tt \scriptsize
  \color{foreground} https://github.com/aaronpeikert/repro-talk}
}


\begin{document}

% title slide
{
{
\setbeamertemplate{footline}{} % no page number here
\frame{
  \titlepage

  \vfill \hfill \includegraphics[height=6mm]{Figs/cc-zero.png} \vspace*{-1cm}

  \note{These Slides are for a medium length talk (30min) meant as an introduction to a reproducible research workflow.

    Source: {\tt https://github.com/aaronpeikert/repro-talk}
}}
}
\begin{frame}[c]
  \begin{center}
  \large
  \textcolor<2>{lolit}{``Insanity is doing the same thing over and over again and expecting different results."}
  \end{center}
  \textcolor<2>{lolit}{\hfill {\textendash} Albert Einstein}\\
  \onslide<2>{
  \begin{center}
  As it turns out doing the\\
    \textcolor{hilit}{{\large same thing}}\\
  is pretty complicated.
\end{center}}

\end{frame}
\begin{frame}[c]{Reproduction ≠ Replication}
  If everything is already there:
\begin{itemize}
  \item published paper
  \item data originally used
  \item code originally used
\end{itemize}
\onslide<2->{Shouldn't that be enough for \textcolor<2>{vhilit}{Reproducibility}? }\onslide<3>{\textcolor{hilit}{Unlikely.}}
\end{frame}

{
  \usebackgroundtemplate{\includegraphics[width=\paperwidth]{Figs/ikea.png}}
  \begin{frame}[plain]
  \end{frame}
}

\begin{frame}[c]{Specify Everything}
  \textcolor<3->{lolit}{The relations between\\
  \textcolor<2>{hilit}{code}, \textcolor<2>{hilit}{data}, \textcolor<2>{hilit}{results} and their \textcolor<2>{hilit}{environment}\\
  need to be \textcolor<2>{vhilit}{unambiguously} specified.\\}
  \vspace{10mm}
  \onslide<3->{\textcolor<4->{lolit}{
  IMHO: Only computer code is unambiguous.\\}}
  \onslide<4>{Spoiler: Computer code is \textcolor{hilit}{not} unambiguous.}
\end{frame}

\begin{frame}[c]{Concepts to fix relations}
  \begin{itemize}
    \item Code \textemdash{} document = literate programming
    \item Code version \textemdash{} data version = version control
    \item Code \textemdash{} interim results = dependency management
    \item Code \textemdash{} execution environment = containerization
  \end{itemize}
\end{frame}

\begin{frame}[c]{Tools for R Users}
	In the R Universe and beyond the most popular are:
  \begin{itemize}
    \item literate programming = RMarkdown*
    \item version control = Git**
    \item dependency management = Make**
    \item containerization = Docker**
  \end{itemize}
  \vfill
  \textcolor{lolit}{
	* RMarkdown supports more then 40 languages e.g.:\\
	\hspace{10mm}Python, Julia, SAS, Scala \& Octave\\
	** Language agnostic
	}
\end{frame}

\begin{frame}[c]{RMarkdown\textemdash{}Literate Programming}
  Text and code are intermingled\\
  into a single source document\\
  that can be \textcolor{hilit}{dynamically} compiled\\
  into various representations:
  \begin{itemize}
    \item (APA conformable) manuscripts
    \item presentations
    \item websites
    \item books
    \item posters
  \end{itemize}
\end{frame}

{
  \usebackgroundtemplate{\includegraphics[width=\paperwidth]{Figs/rmarkdown.png}}
  \begin{frame}[plain]
  \end{frame}
}
{
  \usebackgroundtemplate{\includegraphics[width=\paperwidth]{Figs/rmarkdown-rendered.png}}
  \begin{frame}[plain]
  \end{frame}
}
\begin{frame}[c]{Git/GitHub\textemdash{}Version Control}
  Version control is a system that records changes to a set of files
  over time so that you can recall specific versions later.\\
  \vspace{10mm}
  It guarantees that code and data are exactly the same version as used for
  publication.
\end{frame}
\begin{frame}[c]{Make\textemdash{}Dependency Management}

I would argue that the most important tool for reproducible research is [...] GNU make.\\
\hfill {\textendash} \href{https://kbroman.org/minimal_make/}{Karl Broman}\\

\end{frame}

\begin{frame}[c, fragile]{Make\textemdash{}Dependency Management}

Make is a ``recipe" language that describes how files depend on each other and how to resolve these dependencies.
\vspace{10mm}
\begin{lstlisting}[language=make,basicstyle=\ttfamily\scriptsize]
cfcs-example.pdf: cfcs-example.Rmd data/CFCS.csv
  $(run) Rscript -e 'rmarkdown::render("$(current_dir)/$<")'

data/CFCS.csv: R/00load_data.R
  $(run) Rscript -e 'source("$(current_dir)/$<")'
\end{lstlisting}
\end{frame}

\begin{frame}[c, fragile]{Docker\textemdash{}Containerization}
	Docker is a lightweight virtual computer that is completely independent.\\
	Dockerfiles are ``recipes" that describe what to install on that virtual computer:
	\vspace{10mm}
	\begin{lstlisting}[language=make,basicstyle=\ttfamily\scriptsize]
FROM rocker/verse:3.6.1
ARG BUILD_DATE=2019-11-11
RUN install2.r --error --skipinstalled\
  here lavaan
WORKDIR /home/rstudio
\end{lstlisting}
\end{frame}

\begin{frame}[c]{Future}

\begin{itemize}
	\item The `repro' package shall:
	\begin{itemize}
		\item come soon on: \href{https://github.com/aaronpeikert/repro}{https://github.com/aaronpeikert/repro}
		\item ease the installation process
		\item automatize the setup of new projects
		\item put an end to manual writing of Docker- \& Makefiles
		\item enable automatic execution on GitHub
	\end{itemize}
	\item hands-on workshop
	\begin{itemize}
		\item Where? MPIB Berlin
		\item When? 20.02.20 10h00 \textendash{} 16h00
	\end{itemize}
	\item \textcolor{lolit}{replicable preregistered collaboration project}
\end{itemize}
\vspace{10mm}
Interested?! Write me an e-mail:\\ \hfill \textcolor{lolit}{peikert@mpib-berlin.mpg.de}
\end{frame}

\begin{frame}[c]{Further Reading}
	\begin{itemize}
		\item Preprint of this workflow:\\ \href{https://psyarxiv.com/8xzqy/}{https://psyarxiv.com/8xzqy/}
		\item Testbed for this workflow:\\ \href{https://github.com/aaronpeikert/workflow-showcase}{https://github.com/aaronpeikert/workflow-showcase}
		\item Best talk on the topic:\\ \href{https://github.com/karthik/rstudio2019}{https://github.com/karthik/rstudio2019}
		\item For a deep dive:\\ \href{https://github.com/alan-turing-institute/the-turing-way}{https://github.com/alan-turing-institute/the-turing-way}
	\end{itemize}
\end{frame}

\begin{frame}[c]{One little thing}
	\textcolor{lolit}{It would be great to know from what place you come!}\\
	Could you answer ten questions regarding this talk on:\\
	\centering \large
	\textcolor{hilit}{\href{repro.formr.org}{repro.formr.org}}\\
	\Huge ?
\end{frame}

\end{document}
